\documentclass[12pt]{book}
\usepackage[utf8]{inputenc}
\usepackage{amsmath}
\usepackage{amsfonts}
\usepackage{amssymb}
\usepackage[spanish]{babel}
\usepackage{amsmath}
\usepackage{amssymb,amsfonts,latexsym,cancel}
\begin{document}
\begin{titlepage}
\centering
\bfseries\LARGE(UNIVERSIDAD DEL CAUCA)

\vspace{1cm}
\scshape( FACULTAD INGENIERIA CIVIL)

\vspace{2cm}
\scshape( TEMA :ECUACIONES DIFERENCIALES)

\vspace{2cm}
\large(presenta: JUAN JOSE ZAPATA RANGEL)

\vspace{2 cm} 
\large(presentado a : jhonatan C.)

\vspace{0.5cm}
\large(fecha:29/08/2022)

\end{titlepage}
\begin{center}
INTRODUCCION 

en este documento realizado con un codigo en latex vamos a encontrar un breve resumen de lo que son las ecuaciones diferenciales asi como el tema relacionado con modelos lineales/sistema de resorte de masas econtraremos un resumen de este tema asi como un ejercicio resuelto ,sera una experiencia en la cual aprenderemos a realizar trabajos con esta excelente erramienta.
\end{center}
\begin{center}
\chapter{•ecuaciones diferenciales}
\end{center}
las ecuaciones diferenciales son una ecuación que involucra a las derivadas de una función con la propia función y/o las variables de las que depende. En sus aplicaciones, las funciones generalmente representan cantidades y las derivadas son las tasas de variación de estas cantidades.la ecuación que relaciona diferentes funciones de cambio y da como resultado otra función sería una ecuación diferencial.
\section{temas que me toco investigar}
en el curso de verano que estamos cursando con el profe jhonnatan collazos se asignaron unos temas muy interesantes para que conocieramos mas a fondo sobre las ecuaciones diferenciales, los que me tocaron a mi son los siguientes :
\section{Modelos Lineales : Problemas con valores iniciales }
\subsection*{sistema resorte masa/movimiento libre no amortiguado•} 
Cuando hablamos de movimientos libres no amortiguados significa que no hay ninguna fuerza externa actuando sobre el sistema, no amortiguado significa que no hay nada que le ofrezca resistencia  a la masa que se esta desplazando 
\subsection{LEY DE HOOKE}
 Suponga que un resorte se suspende verticalmente de un soporte

rígido y luego se le fi ja una masa m a su extremo libre. Por supuesto, la cantidad de alar-
gamiento o elongación del resorte depende de la masa; masas con pesos diferentes

alargan el resorte en cantidades diferentes. Por la ley de Hooke, el resorte mismo ejerce

una fuerza restauradora F opuesta a la dirección de elongación y proporcional a la canti-
dad de elongación s y es expresada en forma simple como F  ks, donde k es una constan -

te de proporcionalidad llamada constante de resorte. El resorte se caracteriza en esen-
cia por el número k. Por ejemplo, si una masa que pesa 10 libras hace que un resorte se

alargue $\frac{1}{•2} pie$, entonces $10 = k \frac{•1}{•2}$  implica que $ k = \frac{lib}{pie}.$ Entonces necesariamente

una masa que pesa, digamos, 8 libras alarga el mismo resorte sólo $\frac{2}{5}$pie.
\subsection{SEGUNDA LEY DE NEWTON}
 Después de que se une una masa m a un resorte, ésta alarga el resorte una cantidad s y logra una posición de equilibrio en la cual su peso W se
equilibra mediante la fuerza restauradora ks. Recuerde que el peso se defi ne mediante
W  mg, donde la masa se mide en slugs, kilogramos o gramos y $g= \frac{32pie}{seg^{2}}$, $\frac{9,8m}{seg^{2}}$,o bien $\frac{908cm}{seg^{2}}$, respectivamente. la condición de equilibrio es $mg = ks$ o $mg  ks = 0$. Si la masa se desplaza por una cantidad x de su po-
sición de equilibrio, la fuerza restauradora del resorte es entonces $k(x + s)$. Suponiendo
que no hay fuerzas restauradoras que actúan sobre el sistema y suponiendo que la masa
vibra libre de otras fuerzas externas movimiento libre se puede igualar la segunda
ley de Newton con la fuerza neta o resultante de la fuerza restauradora y el peso.
\section{ejemplo}
Una masa que pesa 2 libras alarga 6 pulgadas un resorte. En $t = 0$ se libera la masa desde un punto que está 8 pulgadas abajo de la posición de equilibrio con una velocidad ascendente de $\frac{4}{•3} \frac{pies}{•s}$. Determine la ecuación de movimiento.

\subsection{•solucion} 
 Debido a que se está usando el sistema de unidades de ingeniería, las mediciones dadas en términos de pulgadas se deben convertir en pies:
\[ 6 pulg = \frac{1}{•2} pie ; 8 pulg =  \frac{2}{•3} pie.
 \]
 Además, se deben convertir las unidades de peso dadas en libras aunidades de masa.
  De m  Wg tenemos que 
  
 \[ m= \frac{2}{32•} =\frac{1}{•16} slug.
  \] 
   También, de la ley de Hooke,
   \[ 2= k \frac{1}{•2} 
   \] implica que la constante de resorte es \[k = 4 \frac{lib}{pie}.
   \] Por lo que, de la ecuación (1) se obtiene
  \[ \frac{1}{16} \frac{d^{2}x}{dt^{2}•}= -4x o \frac{d^{2}x}{dt^{2}•} + 64x = 0
  \]
El desplazamiento inicial y la velocidad inicial son
\[ x(0)=\frac{2}{•3},x^{,}(0)=\frac{-4}{•3} 
  \]donde el signo negativo en la última condición es una consecuencia del hecho de que a la masa 
se le da una velocidad inicial en la dirección negativa o hacia arriba.
ahora
 \[ w^{2}= 64 0  w= 8
 \]
  por lo que la solucion de la ecuacion diferencial es 
\[x(t)= c1 cos8t + c2 sen8t   
\]
Aplicando las condiciones iniciales a x(t) y x'(t) se obtiene 
\[c1= \frac{2}{•3} y c2= \frac{-1}{•6}
 \]
por lo tanto la ecuacion de movimiento es
 \[x(t)= \frac{2}{•3} cos8t + \frac{-1}{•6} sen8t\]
 
 
 
 
 
\begin{center}
\subsection{conclusiones}
al final de este documento lo gramos concluir que esta herramienta llamada latex nos facilita la realizacion de documentos matematicos , ahora hablando un poco del tema y lo que son las ecuaciones diferenciales pudimos complementar las informaciones que nos dio el profe jhonnatan y entender lo que podemos hacer utilizando estas herramientas.

\subsection{bibliografia}
-libro: ecuaciones diferenciales con problemas con valores en la frontera 7ma edicion.

\end{center}




\end{document}